\section*{Results and Analysis}
\hspace{\parindent}In the modeling section, we referred to two models to be studied: one was ideal and was only to understand the pure core of the Corona discharge phenomenon and how to exploit it well, and the other was the realistic one, from which all our results and analysis will diverge.
\subsection*{The model of wire to cylinder lifter }
\hspace{\parindent}In our analysis of this model, we took four different methods into consideration and aimed to compare the results between them in order to get a good sense of the so-far accurate details of how the lifter works.\\
The scenarios we adopted to get the results were based on four pillars, as follows:\\
\textbf{1.	Hardware implementation of the lifter}\\
The objective was to implement the lifter physically and obtain accurate results by measuring the thrust generated.\\
\textbf{2. Simulating the model using Comsol	 }\\
This method involved utilizing numerical methods to solve the equations adopted by the model, enabling the accurate determination of thrust and the plotting of results.\\
 \textbf{3.	Simulating the model using CST}\\
 This method relied on predefined software packages designed for simulating corona discharge. The focus was on simulating the model using these tools and parameters.\\
 \textbf{4.	Solving the equations using Matlab}\\
 This method closely resembles the previous one, as it aimed to employ multiple simulation tools, including Comsol, to address the challenges encountered during the simulation process.\\
 
